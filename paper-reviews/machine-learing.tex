%\documentclass[10pt,twocolumn,letterpaper,draft]{article}
\documentclass[10pt,letterpaper]{article}

% 使用中文宏包
\usepackage[UTF8]{ctex}
\usepackage{graphicx} %插入图片的宏包
\usepackage{float} %设置图片浮动位置的宏包
\usepackage{subfigure} %插入多图时用子图显示的宏包
%\usepackage[utf8]{inputenc}
\usepackage[strings]{underscore}
\usepackage{times}
\usepackage{epsfig}
\usepackage{amsmath}
\usepackage{amssymb}
\usepackage{overpic}
\usepackage{listings}
\usepackage{color}
\usepackage{enumitem}
\setenumerate[1]{itemsep=0pt,partopsep=0pt,parsep=\parskip,topsep=5pt}
\setitemize[1]{itemsep=0pt,partopsep=0pt,parsep=\parskip,topsep=5pt}
\setdescription{itemsep=0pt,partopsep=0pt,parsep=\parskip,topsep=5pt}

\definecolor{mygreen}{rgb}{0,0.6,0}
\definecolor{mygray}{rgb}{0.5,0.5,0.5}
\definecolor{mymauve}{rgb}{0.58,0,0.82}
\lstset{ %
  backgroundcolor=\color{white},   % choose the background color
  basicstyle=\footnotesize,        % size of fonts used for the code
  breaklines=true,                 % automatic line breaking only at whitespace
  captionpos=b,                    % sets the caption-position to bottom
  commentstyle=\color{mygreen},    % comment style
  escapeinside={\%*}{*)},          % if you want to add LaTeX within your code
  keywordstyle=\color{blue},       % keyword style
  stringstyle=\color{mymauve},     % string literal style
}

% Include other packages here, before hyperref.

% If you comment hyperref and then uncomment it, you should delete
% egpaper.aux before re-running latex.  (Or just hit 'q' on the first latex
% run, let it finish, and you should be clear).
\usepackage[pagebackref=true,breaklinks=true,letterpaper=true,colorlinks,bookmarks=false]{hyperref}


\def\httilde{\mbox{\tt\raisebox{-.5ex}{\symbol{126}}}}

\newcommand{\cmm}[1]{\textcolor[rgb]{0,0.6,0}{CMM: #1}}
\newcommand{\todo}[1]{{\textcolor{red}{\bf [#1]}}}
\newcommand{\alert}[1]{\textcolor[rgb]{.6,0,0}{#1}}

\newcommand{\IT}{IT\cite{98pami/Itti}}
\newcommand{\MZ}{MZ\cite{03ACMMM/Ma_Contrast-based}}
\newcommand{\GB}{GB\cite{conf/nips/HarelKP06}}
\newcommand{\SR}{SR\cite{07cvpr/hou_SpectralResidual}}
\newcommand{\FT}{FT\cite{09cvpr/Achanta_FTSaliency}}
\newcommand{\CA}{CA\cite{10cvpr/goferman_context}}
\newcommand{\LC}{LC\cite{06acmmm/ZhaiS_spatiotemporal}}
\newcommand{\AC}{AC\cite{08cvs/achanta_salient}}
\newcommand{\HC}{HC-maps }
\newcommand{\RC}{RC-maps }
\newcommand{\Lab}{$L^*a^*b^*$}
\newcommand{\mypara}[1]{\paragraph{#1.}}

\graphicspath{{figures/}}

\setcounter{page}{1}

\begin{document}
%\begin{CJK*}{GBK}{song}
 

%%%%%%%%% TITLE

\title{论文阅读笔记}

\author{纳文琪$^{1}$}

\maketitle


\section{学习算法的性能度量\cite{2016机器学习}}
\subsection{错误率和精度}
\paragraph{错误率} 指的是分类错误的样本占总样本数的比例,主要适用于二分类问题,也可用于多分类问题。
\paragraph{精度} 指的是分类正确的样本数占总样本数的比例,同意适用于二分类和多分类问题。
\paragraph{} 错误率和精度简单、常用,但并不能满足所有需求。

\subsection{查准率、查全率和F1}
\paragraph{混淆矩阵} 对二分类问题,可以将真实类别与预测类别组合划分成TP、TN、FP、FN四种情形,分别表示预测正确的正例和反例、预测错误的
正例和反例。分类结果可以使用一个“混淆矩阵”表示:\\
\begin{figure}[H] %H为当前位置,!htb为忽略美学标准,htbp为浮动图形
	\centering %图片居中
	\includegraphics[width=0.7\textwidth]{../images/confusion_matrix.png} %插入图片,[]中设置图片大小,{}中是图片文件名
	\caption{Confusion Matrix} %最终文档中希望显示的图片标题
	\label{Fig.main2} %用于文内引用的标签
\end{figure}

\paragraph{查准率(准确率,precision)} 是从预测结果(其数量作为分母)出发计算的精度,指的是预测为正例的样本中,有多少的预测正确的。其定义是:\\
\begin{equation}
	P = \frac{TP}{TP + FP}
\end{equation}

\paragraph{查全率(召回率,recall)} 是从样本(其数量作为分母)出发计算的精度,指的是所有正例样本中,有多少被预测正确了。其定义是:\\
\begin{equation}
	R = \frac{TP}{TP + FN}
\end{equation}

\paragraph{} 查全率与查准率是一对矛盾的度量。

\paragraph{P-R曲线} 用于直观地显示学习器在样本总体上的查全率、查准率。

\paragraph{F1和$F_\beta$} F1是综合考虑查全率和查准率的度量,定义为:\\
\begin{equation}
	F1 = \frac{ 2 \times P \times R }{P + R}
\end{equation}
一些应用中,对查准率和查全率的重视程度不同,此时需要用$F_\beta$ : \\
\begin{equation}
	F_\beta = \frac{ (1+\beta^2) \times P \times R }{(\beta^2 \times P) + R}
\end{equation}
$F_\beta$是F1的一般形式,当$\beta = 1$时就是F1;当$\beta > 1$时,查全率有更大影响;当$\beta < 1$时,查全率有更大影响。

\subsection{ROC与AUC}
\paragraph{ROC} 全称是“受试者工作特征”,与P-R曲线类似,它也有两个坐标,其纵坐标表示的是“真正例率”(TPR),即正例的查全率,横坐标表示的是“假正例率”,及反例被判断错误的比率。两者定义为:\\
\begin{equation}
	TPR = \frac{TP}{TP + FN}
\end{equation}
\begin{equation}
	FPR = \frac{FP}{TN + FP}
\end{equation}
ROC曲线的对角线对应的是“随机猜想”模型,而点(0,1)则代表“理想模型”。

\paragraph{AUC} 进行学习器比较时,与P-R曲线类似,若一个学习器的ROC曲线被另一个学习器的曲线完全包住,则可断言后者的性能优于前者;若两个学习器的ROC曲线有交叉,则需要比较ROC曲线下的面积,即AUC来进行判断。



\subsection{Deep Neural Networks for Bot Detection\cite{kudugunta2018bot}}

\paragraph{Motivation} 现有的系统都是在account-level进行bots的发现,需要根据特定帐号的一系列历史活动记录来确认帐号是否是bot。这在进行检测的时候代价非常昂贵。论文希望通过仅仅一条tweet来判断是否是bot。

\paragraph{} 论文将bot发现的方法分为用户级和tweet级,进行分类;tweet级根据
\paragraph{数据集} 论文使用的数据集存在不平衡问题,作者分别使用SMOTE+ENN和SOMTE+TOMEK的方法平衡数据。数据分为用户级和tweet级两类,用户级数据(用户元数据)包括statuses count、followers count等,tweet级数据包括retweet count、number of hashtags等。


\bibliographystyle{ieeepes}
\bibliography{../Saliency}
\end{document}



























































