%\documentclass[10pt,twocolumn,letterpaper,draft]{article}
\documentclass[10pt,letterpaper]{article}

% 使用中文宏包
\usepackage[UTF8]{ctex}
\usepackage[strings]{underscore}
\usepackage{times}
\usepackage{epsfig}
\usepackage{amsmath}
\usepackage{amssymb}
\usepackage{overpic}
\usepackage{listings}
\usepackage{color}
\usepackage{authblk}
\usepackage{enumitem}
\setenumerate[1]{itemsep=0pt,partopsep=0pt,parsep=\parskip,topsep=5pt}
\setitemize[1]{itemsep=0pt,partopsep=0pt,parsep=\parskip,topsep=5pt}
\setdescription{itemsep=0pt,partopsep=0pt,parsep=\parskip,topsep=5pt}

% 添加首行缩进,两个字符
\usepackage{indentfirst}
\setlength{\parindent}{4em}

\usepackage[pagebackref=true,breaklinks=true,letterpaper=true,colorlinks,bookmarks=false]{hyperref}


\def\httilde{\mbox{\tt\raisebox{-.5ex}{\symbol{126}}}}

\setcounter{page}{1}

\begin{document}
 

%%%%%%%%% TITLE

\title{网络空间安全视角下的新时代中国与当代世界关系}

\author[*]{纳文琪}
\affil[*]{信息学院,学号:22018000164}
\date{}
\maketitle



\section{全球网络空间安全形势}



\paragraph{} 当今社会,计算机及网络技术不断发展,已经渗透到了经济、文化、教育、科研和社会生活的各个领域。
网络已经成为人们日常生活和企业、政府实施管理不可或缺的一个部分,很大程度上改变了人类生存、生活和社会生产组织模式。
\paragraph{} 当今世界,全球网民覆盖率已超过50\%,中国网民规模达到了7.72亿\cite{2018中国互联网络发展状况统计报告} 。互联网的快速发展给当今社会带来了巨大的资源膨胀,促使全球云计算市场快速增长,大量的应用服务器集中使用云计算资源。到202年将会达到200亿美元的市场规模。这对传统网络架构和服务模式的安全保障体系提出了更高的要求和更大的挑战。
\paragraph{} 随着各类资源大量接入网络,特别是物联网技术的快速发展,使得大量物件接入网络,各类软件漏洞、恶意程序、木马病毒、僵尸病毒网络也随之而来,对全球经济发展造成了极大的冲击。
\paragraph{} 什么是网络空间安全?文献\cite{方滨兴2018定义网络空间安全} 基于不同应用需求及研究领域,将网络空间安全定义到载体、资源、主体和操作上,并将其分为四个层:物理安全、运行安全、数据安全和内容安全。
\paragraph{} 2018年以来,国内外影响深远的网络攻击事件频发,举例如下:
\begin{itemize}
	\item 英特尔芯片泄露事件。今年1月,由于英特尔芯片设计的缺陷产生了一个巨大漏洞,导致包括Windows、Linux在内的各大操作系统重新设计系统内核,以规避此漏洞,影响了全球数亿客户端,造成了数十亿美元的经济损失。
	\item 僵尸网络HNS感染逾2万物联网摄像头。随着物联网技术的高速发展,摄像头、传感器等大量安全防护水平低的物联网设备成为了僵尸网络感染的首要目标。今年年初,一种使用定制的点对点通信来诱捕新的物联网设备并构建其基础设施的僵尸程序开始蔓延,此僵尸程序可以通过与Reaper相同的漏洞(CVE-2016-10401和其他针对网络设备的漏洞),对一系列设备进行Web开发,HNS还可以执行多个命令,包括数据泄露、代码执行和对设备操作的干扰。
	\item Facebook爆出史上最大数据泄露丑闻。2018年3月,SCL和剑桥分析公司利用了Facebook上的5000万用户的个人资料数据的丑闻事件爆出。丑闻爆发后,Facebook股价应声大跌,市值大幅度缩水,欧盟、美国、英国纷纷抨击Facebook和剑桥分析公司,Facebook遭遇史上最大公关危机。
\end{itemize}
网络空间安全面临严峻的形式,引发了全球各个国家政府的高度重视。


\paragraph{} 2014年,奥巴马总统宣布启动美国《网络安全框架》部署强化了美国网络安全体系,截止目前,美国已颁布了相关文件40余件。
日本和印度也在积极开展相关工作。日本与2013年6月就发布了《网络安全战略》,明确提出了“网络安全立国”。印度也在同一时间出台了《国家网络安全策略》。
\paragraph{} 我国政府也在积极应对网络安全形势,于2017年6月正式颁布了《中华人民共和国网络安全法》,框架性地构建了许多法律制度和要求,内容涵盖网络信息、内容管理制度、网络安全等级保护制度、关键信息基础设施安全 保护制度 、网络安全审查、个人信息和重要数据保护制度、数据出境安全评估 、网络关键设备和网络安全专用产品安全管理制度、网络安全事件应对制度等。
\paragraph{} 总的来说,越来越多的国家和地区发布了国家网络空间安全相关战略,并构建了相关的执行计划,网络空间安全以及成为了一个国家安全稳定的重要部分。其重要性正随着全球信息化的步伐变得越来越显著。

\section{网络空间安全在当代世界关系中的地位}

\paragraph{} 网络空间安全从来不是一个国家内部的事情,而是和国家以及其在世界关系中所处的地位有及其重要的关系。2013年,美国“棱角门”时间爆发,再一次提高了网络空间安全在国家安全战略层面的高度。在网络空间安全处于一超多强的形式下,网络安全自主可控成为各国安全战略的重要选择,各国都在强调网络安全技术的自主研发,信息安全已经成为了全球化的重要考量。
\paragraph{} 特别在“棱镜门”事件爆发后,引起了一起复杂的国际政治时间。其不仅影响到中美关系,还影响到美俄、美国与欧盟等等各方的关系和利益。其中既包括国家安全层面的问题,也包括个人隐私保护方面的问题。这使得网络空间安全成为了世界多极化的重要博弈,当代国际关系在网络空间安全大背景下,呈现出各种特点。

\section{网络空间安全视角下的新时代中国与当代世界关系}

\paragraph{} 在这样的形势和大背景下,中国应当从网络空间安全的视角出发,考虑应对新时代中国与当代世界的关系。
\paragraph{} 信息技术超级强国利用自身所拥有的技术和资源,对发展中国家进行技术封锁和贸易保护,西方信息技术强国通过实施技术封锁和贸易保护对其进行中国等发展中国家进行打印,例如,美司法部以网络窃密为由起诉中国军方人员,美国大规模拒绝中国信息技术相关留学生的留学签证,对中兴和华为等大型中国信息技术企业进行合规性审查等。
\paragraph{} 为应对新形势下的中国与当地世界关系,中国政府应进一步加强信息安全的战略谋划。自2013年起,中国已在此方面进行了一系列动作。2013年11月,中共中央提出了设计国家安全委员会,完善国家安全体系和国家安全战略,在网络空间安全的战略理念认知和管理范式转型上迈出了极为重要的一步。2014年2月,国家网络与信息化领导小组成立,提出了一体之两翼,双轮驱动的网络空间安全顶层设计。习近平主席同时指出“没有网络安全就没有国家安全,没有信息化就没有现代化”。
\paragraph{} 多年来,国家各级政府、企业都投入大量人财物在网络空间安全体系建设上,从物理安全、运行安全、数据安全等哥哥角度全方位指定了相关法律法规、行业标准等。在网络空间安全的大背景下,简单依靠枪炮保障安全的时代已经过去,网络空间安全正在成为更重要的安全保障手段。只有做好网络空间安全,新时代中国才能在当代世界关系框架下平稳、快速发展,走出一条共同安全、综合安全和可持续安全的新路。

\bibliographystyle{ieeepes}
\bibliography{../Saliency}
\end{document}



























































